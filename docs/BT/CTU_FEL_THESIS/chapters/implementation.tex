\chapter{Implementation}
\label{ch:implementation}

This chapter details the implementation of all system components, including hardware sensor nodes, software simulator, MQTT broker configuration, and integration with Home Assistant.

% ============================================================================
% 4.1 Hardware Implementation
% ============================================================================
\section{Hardware Implementation}
\label{sec:hardware}

Two ESP32-based sensor nodes were implemented to demonstrate different IoT paradigms: edge intelligence and traditional environment sensing. The ESP32 platform was selected for its combination of dual-core processing, WiFi connectivity, and extensive peripheral support~\cite{esp32_datasheet_2024}.

% ----------------------------------------------------------------------------
\subsection{ESP32-S3 Edge Intelligence Node}
\label{subsec:esp32s3}

The ESP32-S3 node serves as an edge computing platform, performing local signal processing to reduce cloud dependency and enable real-time responses~\cite{esp32s3_datasheet_2024}.

\begin{table}[htbp]
\centering
\caption{ESP32-S3 Node Sensor Configuration}
\label{tab:esp32s3-sensors}
\begin{tabular}{llll}
\toprule
\textbf{Sensor} & \textbf{Model} & \textbf{Measurement} & \textbf{Interface} \\
\midrule
Barometric & BMP180 & Temperature, Pressure & I2C \\
Motion & MPU6050 (GY-521) & 6-axis Accelerometer/Gyro & I2C \\
Acoustic & KY-037 & Sound Level & ADC \\
\bottomrule
\end{tabular}
\end{table}

The ESP32-S3's vector instruction set enables efficient implementation of edge computing algorithms:

\begin{itemize}
    \item \textbf{Resultant G-Force:} Calculated as $\sqrt{a_x^2 + a_y^2 + a_z^2}$ from accelerometer readings, enabling vibration magnitude detection.
    
    \item \textbf{Dynamic Vibration Component:} Computed as $|G - 1.0|$ to isolate dynamic motion from static gravitational acceleration.
    
    \item \textbf{Acoustic Peak Detection:} Binary sensor triggered when ADC readings exceed configurable threshold, indicating significant sound events.
\end{itemize}

% TODO: Add code snippet showing edge computing lambda in ESPHome

% ----------------------------------------------------------------------------
\subsection{ESP32 Environment Sensing Node}
\label{subsec:esp32}

The standard ESP32 node focuses on environmental monitoring and safety detection with local automation capabilities.

\begin{table}[htbp]
\centering
\caption{ESP32 Node Sensor Configuration}
\label{tab:esp32-sensors}
\begin{tabular}{llll}
\toprule
\textbf{Sensor} & \textbf{Model} & \textbf{Measurement} & \textbf{Interface} \\
\midrule
Color/Light & TCS34725 & RGB, Lux & I2C \\
Gas & MQ-2 & Smoke/Combustible gas & ADC \\
Motion & SR602 & PIR presence & GPIO \\
Indicator & RGB LED & Status feedback & PWM \\
\bottomrule
\end{tabular}
\end{table}

Local automation rules execute directly on the device without requiring network connectivity:

\begin{itemize}
    \item \textbf{Smoke Alert:} When MQ-2 sensor voltage exceeds 2.5V, the RGB LED activates red pulsing effect and publishes alert message to MQTT.
    
    \item \textbf{Occupancy Indication:} SR602 motion detection triggers green LED illumination for visual feedback.
\end{itemize}

% ----------------------------------------------------------------------------
\subsection{Sensor Integration Details}
\label{subsec:sensor-details}

Both nodes employ the dual-connection architecture described in Section~\ref{sec:comm-protocols}:

\begin{verbatim}
# ESPHome MQTT configuration (excerpt)
mqtt:
  broker: mqtt.example.com
  port: 8883
  certificate_authority: !secret ca_cert
  client_certificate: !secret client_cert
  client_key: !secret client_key
  discovery: true
  discovery_prefix: homeassistant
\end{verbatim}

The \texttt{discovery: true} option enables automatic entity registration in Home Assistant through MQTT discovery protocol.

% TODO: Add Figure showing hardware wiring diagram or photo

% ============================================================================
% 4.2 Software Implementation
% ============================================================================
\section{Software Implementation}
\label{sec:software}

% ----------------------------------------------------------------------------
\subsection{ESPHome Configuration}
\label{subsec:esphome-config}

ESPHome configurations are maintained as YAML files in the project repository. The declarative syntax enables rapid development while supporting custom logic through lambda expressions:

\begin{verbatim}
sensor:
  - platform: template
    name: "Resultant G-Force"
    unit_of_measurement: "g"
    lambda: |-
      float ax = id(accel_x).state;
      float ay = id(accel_y).state;
      float az = id(accel_z).state;
      return sqrt(ax*ax + ay*ay + az*az);
    update_interval: 100ms
\end{verbatim}

This configuration demonstrates edge computing on the ESP32-S3, calculating resultant acceleration locally rather than transmitting raw axis data.

% ----------------------------------------------------------------------------
\subsection{Python Sensor Simulator}
\label{subsec:simulator}

The Python sensor simulator (\texttt{smarthome\_sim} package) enables scalability testing beyond physical hardware limitations. Key features include:

\begin{itemize}
    \item \textbf{Multi-broker Publishing:} Simultaneous connection to multiple MQTT brokers for redundancy testing.
    
    \item \textbf{Configurable Entity Models:} Support for various sensor behaviors including drift, uniform random, and sinusoidal patterns.
    
    \item \textbf{Home Assistant Discovery:} Automatic entity registration through MQTT discovery payloads.
    
    \item \textbf{mTLS Support:} Full mutual TLS authentication using the same certificate infrastructure as hardware nodes.
    
    \item \textbf{Multi-process Scaling:} Worker processes enable simulation of thousands of concurrent devices.
\end{itemize}

\begin{verbatim}
# Example: Running 12000 simulated devices
python -m smarthome_sim \
  --config brokers.yml \
  --devices 12000 \
  --workers 6 \
  --qos 0
\end{verbatim}

% TODO: Add code snippet showing entity model configuration

% ----------------------------------------------------------------------------
\subsection{EMQX Broker Deployment}
\label{subsec:emqx}

EMQX was selected as the MQTT broker for its high performance and native mTLS support~\cite{emqx_2024}. The broker runs in a Docker container on a VPS in Nuremberg, Germany:

\begin{verbatim}
# docker-compose.yml excerpt
services:
  emqx:
    image: emqx/emqx:5.3
    ports:
      - "8883:8883"  # MQTT over TLS
    volumes:
      - ./certs:/etc/emqx/certs
      - ./emqx.conf:/opt/emqx/etc/emqx.conf
\end{verbatim}

The listener configuration enforces mTLS for all connections:

\begin{verbatim}
listeners.ssl.default {
  bind = "0.0.0.0:8883"
  ssl_options {
    verify = verify_peer
    fail_if_no_peer_cert = true
    cacertfile = "/etc/emqx/certs/ca.pem"
    certfile = "/etc/emqx/certs/server.pem"
    keyfile = "/etc/emqx/certs/server.key"
  }
}
\end{verbatim}

% ============================================================================
% 4.3 Home Assistant Integration
% ============================================================================
\section{Home Assistant Integration}
\label{sec:ha-integration}

% ----------------------------------------------------------------------------
\subsection{MQTT Discovery}
\label{subsec:mqtt-discovery}

Home Assistant's MQTT discovery feature enables automatic entity registration without manual configuration. Devices publish JSON payloads to designated discovery topics:

\begin{verbatim}
Topic: homeassistant/sensor/esp32_temperature/config
Payload:
{
  "name": "ESP32 Temperature",
  "unique_id": "esp32_temperature",
  "state_topic": "smarthome/esp32/temperature/state",
  "unit_of_measurement": "°C",
  "device_class": "temperature",
  "device": {
    "identifiers": ["esp32_node_b"],
    "name": "ESP32 Environment Node"
  }
}
\end{verbatim}

This mechanism reduces deployment friction and ensures consistent entity configuration across restarts.

% ----------------------------------------------------------------------------
\subsection{Dashboard Configuration}
\label{subsec:dashboard}

Home Assistant dashboards provide real-time visualization of sensor data and control interfaces. Custom cards display:

\begin{itemize}
    \item Sensor readings with historical graphs
    \item Device connectivity status
    \item Automation trigger history
    \item Alert notifications
\end{itemize}

% TODO: Add screenshot of Home Assistant dashboard

% ============================================================================
% 4.4 Data Pipeline
% ============================================================================
\section{Data Pipeline}
\label{sec:data-pipeline}

% ----------------------------------------------------------------------------
\subsection{InfluxDB Storage}
\label{subsec:influxdb}

InfluxDB serves as the time-series database for long-term sensor data retention. Home Assistant's native integration automatically writes entity state changes:

\begin{verbatim}
# configuration.yaml
influxdb:
  host: localhost
  port: 8086
  database: homeassistant
  include:
    domains:
      - sensor
      - binary_sensor
\end{verbatim}

The time-series storage enables historical analysis, trend detection, and anomaly identification across extended time periods.

% ----------------------------------------------------------------------------
\subsection{Grafana Visualization}
\label{subsec:grafana}

Grafana dashboards provide advanced visualization capabilities beyond Home Assistant's built-in features:

\begin{itemize}
    \item Multi-sensor overlay comparisons
    \item Statistical aggregations (mean, percentiles)
    \item Alerting based on metric thresholds
    \item Custom time range analysis
\end{itemize}

% TODO: Add Grafana dashboard screenshot

% ============================================================================
% 4.5 Automation Examples
% ============================================================================
\section{Automation Examples}
\label{sec:automation}

% ----------------------------------------------------------------------------
\subsection{Home Assistant Automations}
\label{subsec:ha-automation}

Native Home Assistant automations handle event-driven responses using YAML configuration:

\begin{verbatim}
automation:
  - alias: "Smoke Detection Alert"
    trigger:
      - platform: numeric_state
        entity_id: sensor.esp32_smoke
        above: 2.5
    action:
      - service: notify.mobile_app_iphone
        data:
          title: "Smoke Detected"
          message: "Smoke level: {{ states('sensor.esp32_smoke') }}V"
\end{verbatim}

This automation demonstrates integration between hardware sensors and mobile notifications.

% ----------------------------------------------------------------------------
\subsection{Node-RED Flows}
\label{subsec:nodered}

Node-RED provides visual flow-based programming for complex automation scenarios. While Home Assistant handles most automation requirements, Node-RED enables:

\begin{itemize}
    \item Complex conditional logic with multiple branches
    \item Data transformation and aggregation
    \item Integration with external APIs and services
    \item Custom function nodes using JavaScript
\end{itemize}

% TODO: Add Node-RED flow screenshot or diagram
