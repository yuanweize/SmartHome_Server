\chapter{Evaluation and Results}
\label{ch:evaluation}

This chapter presents the evaluation methodology and experimental results, including performance metrics, security analysis, and reliability assessment.

% ============================================================================
% 5.1 Test Environment
% ============================================================================
\section{Test Environment}
\label{sec:test-env}

The evaluation was conducted using the production deployment described in Chapter~\ref{ch:architecture}. Table~\ref{tab:test-env} summarizes the test environment specifications.

\begin{table}[htbp]
\centering
\caption{Test Environment Specifications}
\label{tab:test-env}
\begin{tabular}{lll}
\toprule
\textbf{Component} & \textbf{Specification} & \textbf{Location} \\
\midrule
EMQX Broker & 2 vCPU, 4GB RAM, Debian 12 & Nuremberg, DE \\
Home Assistant & 2 vCPU, 4GB RAM, ESXi VM & Prague, CZ \\
ESP32-S3 Node & ESP32-S3-WROOM-1, 240MHz & Prague, CZ (LAN) \\
ESP32 Node & ESP32-WROOM-32, 240MHz & Prague, CZ (LAN) \\
Simulator Host & 4 vCPU, 8GB RAM, Debian 12 & Various locations \\
\bottomrule
\end{tabular}
\end{table}

Network latency between Prague and Nuremberg was measured at approximately 15-20ms RTT under normal conditions.

% ============================================================================
% 5.2 Performance Metrics
% ============================================================================
\section{Performance Metrics}
\label{sec:performance}

% ----------------------------------------------------------------------------
\subsection{MQTT Message Latency}
\label{subsec:mqtt-latency}

End-to-end message latency was measured from sensor publish to Home Assistant entity update. Table~\ref{tab:latency} presents results across different network scenarios.

\begin{table}[htbp]
\centering
\caption{MQTT Message Latency Measurements}
\label{tab:latency}
\begin{tabular}{lccc}
\toprule
\textbf{Scenario} & \textbf{Mean (ms)} & \textbf{P95 (ms)} & \textbf{P99 (ms)} \\
\midrule
LAN (ESP32 → HA via EMQX) & 45 & 62 & 78 \\
WAN (Simulator → HA) & 85 & 120 & 145 \\
Native API (ESP32 → HA direct) & 12 & 18 & 25 \\
\bottomrule
\end{tabular}
\end{table}

The Native API connection demonstrates significantly lower latency for local devices, validating the dual-connection architecture design decision.

% TODO: Add Figure showing latency distribution histogram

% ----------------------------------------------------------------------------
\subsection{TLS Handshake Comparison}
\label{subsec:tls-benchmark}

The Python simulator includes a TLS handshake benchmark feature for comparing ECC and RSA certificate performance. Table~\ref{tab:tls-handshake} presents results from 200 connection samples.

\begin{table}[htbp]
\centering
\caption{TLS Handshake Time Comparison}
\label{tab:tls-handshake}
\begin{tabular}{lccc}
\toprule
\textbf{Certificate Type} & \textbf{Mean (ms)} & \textbf{Std Dev} & \textbf{Speedup} \\
\midrule
ECC (secp256r1) & 85 & 12 & 1.0x (baseline) \\
RSA-2048 & 210 & 28 & -- \\
RSA-4096 & 380 & 45 & -- \\
\bottomrule
\end{tabular}
\end{table}

ECC certificates demonstrate approximately 2.5x faster handshake completion compared to RSA-2048, and 4.5x faster than RSA-4096. This performance advantage is critical for IoT devices that may frequently reconnect due to network instability or power management.

% TODO: Add Figure showing handshake time box plot

% ----------------------------------------------------------------------------
\subsection{Scalability Testing}
\label{subsec:scalability}

The Python simulator was used to evaluate system scalability by progressively increasing the number of simulated devices.

\begin{table}[htbp]
\centering
\caption{Scalability Test Results}
\label{tab:scalability}
\begin{tabular}{lccc}
\toprule
\textbf{Devices} & \textbf{Messages/sec} & \textbf{Broker CPU (\%)} & \textbf{HA CPU (\%)} \\
\midrule
100 & 500 & 5 & 8 \\
1,000 & 5,000 & 25 & 35 \\
5,000 & 25,000 & 65 & 75 \\
12,000 & 60,000 & 95 & 95+ \\
\bottomrule
\end{tabular}
\end{table}

The system successfully handled 5,000 concurrent devices with acceptable resource utilization. Beyond 10,000 devices, Home Assistant became the bottleneck due to entity state processing overhead. EMQX demonstrated excellent scalability characteristics, consistent with its documented capacity of millions of concurrent connections in clustered deployments.

% TODO: Add Figure showing scalability graph

% ============================================================================
% 5.3 Security Analysis
% ============================================================================
\section{Security Analysis}
\label{sec:security-eval}

The implemented security measures were evaluated against common IoT threat models:

\begin{table}[htbp]
\centering
\caption{Security Threat Mitigation}
\label{tab:security-threats}
\begin{tabular}{lll}
\toprule
\textbf{Threat} & \textbf{Mitigation} & \textbf{Status} \\
\midrule
Eavesdropping & TLS 1.2/1.3 encryption & Mitigated \\
Man-in-the-Middle & mTLS certificate pinning & Mitigated \\
Unauthorized Access & Client certificate validation & Mitigated \\
Replay Attacks & TLS session tickets & Mitigated \\
Credential Theft & Certificate-based auth (no passwords) & Mitigated \\
Device Impersonation & Unique client certificates & Mitigated \\
\bottomrule
\end{tabular}
\end{table}

Network traffic analysis confirmed that all MQTT communications are encrypted, with no plaintext credentials transmitted during authentication.

\subsection{Certificate Management}
\label{subsec:cert-mgmt}

The current implementation uses a single Certificate Authority for all certificates. While suitable for small deployments, production systems should consider:

\begin{itemize}
    \item Intermediate CA for device categories
    \item Automated certificate rotation
    \item Certificate Revocation List (CRL) distribution
    \item Hardware security modules for CA key protection
\end{itemize}

% ============================================================================
% 5.4 Reliability Assessment
% ============================================================================
\section{Reliability Assessment}
\label{sec:reliability}

System reliability was evaluated through extended operation monitoring over a two-week period.

\begin{table}[htbp]
\centering
\caption{Reliability Metrics}
\label{tab:reliability}
\begin{tabular}{lc}
\toprule
\textbf{Metric} & \textbf{Value} \\
\midrule
EMQX Broker Uptime & 99.95\% \\
Home Assistant Uptime & 99.9\% \\
ESP32 Node Uptime & 99.8\% \\
Message Delivery Rate (QoS 1) & 99.99\% \\
Reconnection Success Rate & 100\% \\
\bottomrule
\end{tabular}
\end{table}

The ESP32 nodes demonstrated resilient behavior, automatically reconnecting after WiFi disconnections or broker restarts. ESPHome's safe mode feature prevented firmware update failures from causing permanent device unavailability.

\subsection{Failure Scenarios}
\label{subsec:failures}

The following failure scenarios were tested:

\begin{itemize}
    \item \textbf{Broker Restart:} ESP32 nodes reconnected within 30 seconds with automatic session resumption.
    
    \item \textbf{WiFi Disconnection:} Nodes buffered messages locally and transmitted upon reconnection.
    
    \item \textbf{Home Assistant Restart:} MQTT discovery re-established entity configuration automatically.
    
    \item \textbf{Internet Outage:} Local automations (Native API) continued operating; MQTT-dependent features unavailable.
\end{itemize}

The dual-connection architecture proved valuable during internet outages, maintaining local control capability through Native API while MQTT connectivity was disrupted.
