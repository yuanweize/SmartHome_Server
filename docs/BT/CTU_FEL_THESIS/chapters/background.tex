\chapter{Theoretical Background}
\label{ch:background}

This chapter provides the theoretical foundation for understanding smart home systems, communication protocols, and security mechanisms employed in this thesis.

% ============================================================================
% 2.1 Smart Home Systems Overview
% ============================================================================
\section{Smart Home Systems Overview}
\label{sec:smarthome-overview}

A smart home system consists of interconnected devices that monitor and control various aspects of a residential environment. These systems typically comprise three main layers:

\begin{itemize}
    \item \textbf{Perception Layer:} Sensors and actuators that interact with the physical environment, including temperature sensors, motion detectors, smart switches, and lighting controls.
    
    \item \textbf{Network Layer:} Communication infrastructure that enables data transmission between devices, including protocols such as WiFi, Zigbee, Z-Wave, and Bluetooth.
    
    \item \textbf{Application Layer:} Software platforms that process sensor data, execute automation rules, and provide user interfaces for monitoring and control.
\end{itemize}

The evolution of smart home technology has been driven by advances in microcontroller capabilities, wireless communication standards, and cloud computing infrastructure~\cite{smarthome_survey_2020}. Modern systems increasingly emphasize edge computing, where data processing occurs locally rather than in centralized cloud servers~\cite{edge_computing_iot_2016}.

The architecture of IoT systems has been extensively studied, with researchers proposing various layered models to describe the interactions between physical devices, communication networks, and application services~\cite{iot_architecture_2019}.

% ============================================================================
% 2.2 MQTT Protocol
% ============================================================================
\section{MQTT Protocol}
\label{sec:mqtt}

Message Queuing Telemetry Transport (MQTT) is a lightweight publish-subscribe messaging protocol designed for constrained devices and low-bandwidth networks~\cite{mqtt_oasis_2019}. Originally developed by IBM in 1999, MQTT has become a de facto standard for IoT communications due to its minimal bandwidth requirements and simple implementation~\cite{mqtt_iot_survey_2017}.

Comparative studies have demonstrated MQTT's advantages over HTTP for IoT applications, particularly in terms of reduced network overhead and power consumption~\cite{mqtt_performance_2018}.

% ----------------------------------------------------------------------------
\subsection{Publish/Subscribe Model}
\label{subsec:pubsub}

Unlike traditional request-response protocols, MQTT employs a publish-subscribe model with a central broker:

\begin{itemize}
    \item \textbf{Publishers} send messages to specific topics without knowledge of subscribers.
    \item \textbf{Subscribers} register interest in topics and receive all messages published to those topics.
    \item \textbf{Broker} acts as an intermediary, routing messages from publishers to subscribers based on topic matching.
\end{itemize}

This decoupled architecture enables scalable many-to-many communication patterns suitable for IoT deployments.

% TODO: Add Figure showing MQTT publish-subscribe model

% ----------------------------------------------------------------------------
\subsection{Quality of Service Levels}
\label{subsec:qos}

MQTT defines three Quality of Service (QoS) levels to balance reliability against overhead:

\begin{table}[htbp]
\centering
\caption{MQTT Quality of Service Levels}
\label{tab:mqtt-qos}
\begin{tabular}{cll}
\toprule
\textbf{QoS} & \textbf{Delivery Guarantee} & \textbf{Use Case} \\
\midrule
0 & At most once (fire and forget) & Non-critical sensor readings \\
1 & At least once (acknowledged) & Important state changes \\
2 & Exactly once (four-way handshake) & Critical commands \\
\bottomrule
\end{tabular}
\end{table}

QoS 0 provides the lowest overhead but no delivery guarantees. QoS 1 ensures delivery through acknowledgment but may result in duplicates. QoS 2 guarantees exactly-once delivery at the cost of additional message exchanges.

% ----------------------------------------------------------------------------
\subsection{Topic Hierarchy}
\label{subsec:topics}

MQTT topics use a hierarchical structure with forward slash separators, enabling flexible subscription patterns:

\begin{verbatim}
smarthome/living_room/temperature/state
smarthome/bedroom/+/state        (single-level wildcard)
smarthome/#                      (multi-level wildcard)
\end{verbatim}

This hierarchical organization facilitates logical grouping of devices and supports pattern-based subscriptions for efficient message filtering.

% ============================================================================
% 2.3 Transport Layer Security
% ============================================================================
\section{Transport Layer Security}
\label{sec:tls}

Transport Layer Security (TLS) provides cryptographic security for communications over computer networks. In IoT contexts, TLS protects against eavesdropping, tampering, and impersonation attacks~\cite{iot_security_survey_2019}.

% ----------------------------------------------------------------------------
\subsection{TLS and mTLS}
\label{subsec:mtls}

Standard TLS authenticates only the server to the client, verifying the server's identity through certificate validation. Mutual TLS (mTLS) extends this by requiring both parties to present certificates:

\begin{enumerate}
    \item Client initiates connection with ClientHello message.
    \item Server responds with certificate and requests client certificate.
    \item Client presents its certificate for server verification.
    \item Both parties complete key exchange and establish encrypted session.
\end{enumerate}

mTLS provides stronger security guarantees for IoT deployments by ensuring that only authorized devices can connect to the message broker.

% TODO: Add sequence diagram for mTLS handshake

% ----------------------------------------------------------------------------
\subsection{ECC vs RSA Cryptography}
\label{subsec:ecc-rsa}

Elliptic Curve Cryptography (ECC) offers equivalent security to RSA with significantly smaller key sizes, making it particularly suitable for resource-constrained IoT devices~\cite{ecc_vs_rsa_2020}.

\begin{table}[htbp]
\centering
\caption{Comparison of ECC and RSA Key Sizes}
\label{tab:ecc-rsa}
\begin{tabular}{lcc}
\toprule
\textbf{Security Level} & \textbf{ECC Key Size} & \textbf{RSA Key Size} \\
\midrule
80-bit & 160 bits & 1024 bits \\
112-bit & 224 bits & 2048 bits \\
128-bit & 256 bits & 3072 bits \\
\bottomrule
\end{tabular}
\end{table}

The secp256r1 curve (also known as P-256 or prime256v1) provides 128-bit security with 256-bit keys, representing a 12x reduction compared to equivalent RSA-3072 keys. This results in faster cryptographic operations and reduced memory requirements on microcontrollers.

% ============================================================================
% 2.4 Home Assistant Platform
% ============================================================================
\section{Home Assistant Platform}
\label{sec:homeassistant}

Home Assistant is an open-source home automation platform designed to run on local hardware~\cite{homeassistant_2024}. Key features include:

\begin{itemize}
    \item \textbf{Integration Support:} Over 2,000 integrations for commercial and DIY devices.
    \item \textbf{Local Control:} All processing occurs locally without mandatory cloud connectivity.
    \item \textbf{Automation Engine:} YAML-based or visual automation configuration.
    \item \textbf{Dashboard System:} Customizable user interfaces for monitoring and control.
\end{itemize}

Home Assistant supports MQTT discovery, enabling automatic registration of devices that publish configuration payloads to designated discovery topics.

% ============================================================================
% 2.5 ESPHome Framework
% ============================================================================
\section{ESPHome Framework}
\label{sec:esphome}

ESPHome is a firmware framework for ESP8266 and ESP32 microcontrollers that generates custom firmware from YAML configuration files~\cite{esphome_2024}. Key capabilities include:

\begin{itemize}
    \item \textbf{Sensor Support:} Native drivers for hundreds of sensor types.
    \item \textbf{Native API:} Low-latency direct communication with Home Assistant.
    \item \textbf{MQTT Support:} Full MQTT client with TLS/mTLS capability.
    \item \textbf{OTA Updates:} Over-the-air firmware updates via WiFi.
    \item \textbf{Local Automation:} On-device automation without external dependencies.
\end{itemize}

The declarative YAML configuration approach reduces development complexity while maintaining flexibility through lambda expressions for custom logic.

% ============================================================================
% 2.6 Edge Computing in IoT
% ============================================================================
\section{Edge Computing in IoT}
\label{sec:edge}

Edge computing refers to processing data near its source rather than transmitting all data to centralized cloud servers. Benefits include:

\begin{itemize}
    \item \textbf{Reduced Latency:} Local processing eliminates round-trip delays to remote servers.
    \item \textbf{Bandwidth Efficiency:} Only processed results need transmission.
    \item \textbf{Privacy Enhancement:} Raw sensor data remains on local devices.
    \item \textbf{Offline Capability:} Critical functions operate without internet connectivity.
\end{itemize}

The ESP32-S3 microcontroller features vector instructions optimized for signal processing and machine learning inference, enabling edge computing applications such as vibration analysis and acoustic event detection directly on the sensor node.

% TODO: Add citation for edge computing in IoT survey
