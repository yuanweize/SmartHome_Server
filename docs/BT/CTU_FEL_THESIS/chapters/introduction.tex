\chapter{Introduction}
\label{ch:introduction}

% ============================================================================
% 1.1 Background and Motivation
% ============================================================================
\section{Background and Motivation}
\label{sec:background}

The concept of smart homes has evolved significantly over the past decade, transforming from a futuristic vision into a practical reality adopted by millions of households worldwide. Smart home systems integrate various sensors, actuators, and control platforms to provide automated environmental monitoring, security surveillance, and energy management capabilities.

At the core of modern smart home architectures lie servers and Unix-like operating systems that provide the computational infrastructure for data processing, storage, and decision-making. These systems handle the aggregation of sensor data from potentially hundreds of devices, execute automation rules, and provide user interfaces for monitoring and control.

The proliferation of Internet of Things (IoT) devices has created new challenges in terms of scalability, security, and interoperability~\cite{iot_architecture_2019}. Commercial smart home platforms such as Amazon Alexa, Google Home, and Xiaomi Mi Home offer convenient solutions but often require cloud connectivity and raise concerns about data privacy and vendor dependency~\cite{voice_assistant_privacy_2019}.

% ============================================================================
% 1.2 Problem Statement
% ============================================================================
\section{Problem Statement}
\label{sec:problem}

Despite the availability of numerous commercial smart home solutions, several challenges remain inadequately addressed:

\begin{itemize}
    \item \textbf{Data Sovereignty:} Commercial platforms typically transmit sensor data to cloud servers, limiting user control over personal information and raising privacy concerns.
    
    \item \textbf{Vendor Lock-in:} Proprietary ecosystems restrict device interoperability and create dependency on specific manufacturers, as evidenced by platform discontinuations that render devices unusable~\cite{iot_vendor_lockin_2021}.
    
    \item \textbf{Security Vulnerabilities:} Many IoT devices lack adequate security measures, with unencrypted communications and weak authentication mechanisms being common issues~\cite{iot_security_survey_2019}.
    
    \item \textbf{Scalability Limitations:} Consumer-grade solutions often struggle to handle large numbers of devices or high-frequency sensor data streams.
\end{itemize}

This thesis addresses these challenges by designing and implementing an open-source smart home system that prioritizes local data processing, secure communications, and scalable architecture.

% ============================================================================
% 1.3 Research Objectives
% ============================================================================
\section{Research Objectives}
\label{sec:objectives}

The primary objectives of this thesis are threefold:

\begin{enumerate}
    \item \textbf{Analytical Objective:} Analyze the role of servers and Unix-like systems in smart home sensor monitoring and control, examining the architectural patterns and communication protocols employed in modern IoT deployments.
    
    \item \textbf{Design Objective:} Design and implement a model smart home system using the MQTT protocol and Home Assistant platform, demonstrating integration of both real hardware sensors (ESP32/ESP32-S3) and simulated devices.
    
    \item \textbf{Evaluation Objective:} Compare the implemented solution with commercial alternatives in terms of scalability, latency, privacy, and reliability, providing quantitative metrics and qualitative analysis.
\end{enumerate}

% ============================================================================
% 1.4 Thesis Organization
% ============================================================================
\section{Thesis Organization}
\label{sec:organization}

The remainder of this thesis is organized as follows:

\textbf{Chapter~\ref{ch:background}} provides the theoretical background, covering smart home systems, the MQTT protocol, transport layer security, and the Home Assistant ecosystem.

\textbf{Chapter~\ref{ch:architecture}} presents the system architecture, including network topology, communication protocols, and security design.

\textbf{Chapter~\ref{ch:implementation}} details the implementation of hardware sensor nodes, the Python simulator, and integration with Home Assistant and supporting tools.

\textbf{Chapter~\ref{ch:evaluation}} evaluates the system performance through latency measurements, scalability tests, and security analysis.

\textbf{Chapter~\ref{ch:comparison}} compares the implemented system with commercial smart home platforms, discussing trade-offs between different approaches.

\textbf{Chapter~\ref{ch:conclusion}} concludes the thesis with a summary of contributions, discussion of limitations, and suggestions for future work.
