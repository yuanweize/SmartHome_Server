\chapter*{Abstract}
\addcontentsline{toc}{chapter}{Abstract}

This thesis presents the design and implementation of a smart home sensor control system utilizing servers and Unix-like operating systems. The system integrates real hardware sensors based on ESP32 and ESP32-S3 microcontrollers with a Python-based sensor simulator capable of scaling to over 12,000 virtual devices for stress testing purposes.

The implemented architecture employs MQTT protocol with mutual TLS (mTLS) authentication using ECC certificates for secure communication between distributed components. An EMQX broker hosted in Germany serves as the message hub, while Home Assistant running on an ESXi virtual machine in Prague functions as the central control platform. The ESP32-S3 node demonstrates edge computing capabilities through local vibration analysis and acoustic event detection, leveraging the microcontroller's vector instruction set.

Performance evaluation reveals that ECC-based mTLS reduces TLS handshake time by approximately 2.5 times compared to RSA-2048, making it suitable for resource-constrained IoT devices. Scalability tests confirm the system's ability to handle thousands of concurrent device connections while maintaining acceptable latency.

Comparison with commercial solutions such as Xiaomi Mi Home and Amazon Alexa highlights the trade-offs between ease of use and data sovereignty. The open-source approach provides complete control over data storage and processing, eliminating dependency on vendor services that may be discontinued.

\vspace{5mm}
\noindent\textbf{Keywords:} smart home, MQTT, mTLS, Home Assistant, ESP32, edge computing, IoT security, sensor simulation

% ============================================================================
\newpage
\chapter*{Abstrakt}
\addcontentsline{toc}{chapter}{Abstrakt}

Tato bakalářská práce představuje návrh a implementaci systému pro řízení senzorů v chytré domácnosti s využitím serverů a operačních systémů podobných Unixu. Systém integruje skutečné hardwarové senzory založené na mikrokontrolérech ESP32 a ESP32-S3 s Python simulátorem senzorů, který je schopen škálovat na více než 12~000 virtuálních zařízení pro účely zátěžového testování.

Implementovaná architektura využívá protokol MQTT s oboustrannou autentizací TLS (mTLS) pomocí certifikátů ECC pro bezpečnou komunikaci mezi distribuovanými komponentami. Broker EMQX hostovaný v Německu slouží jako centrum zpráv, zatímco Home Assistant běžící na virtuálním stroji ESXi v Praze funguje jako centrální řídicí platforma. Uzel ESP32-S3 demonstruje možnosti edge computingu prostřednictvím lokální analýzy vibrací a detekce akustických událostí s využitím vektorové instrukční sady mikrokontroléru.

Hodnocení výkonu ukazuje, že mTLS založené na ECC snižuje čas TLS handshake přibližně 2,5krát ve srovnání s RSA-2048, což jej činí vhodným pro IoT zařízení s omezenými zdroji. Testy škálovatelnosti potvrzují schopnost systému zvládnout tisíce současných připojení zařízení při zachování přijatelné latence.

Srovnání s komerčními řešeními jako Xiaomi Mi Home a Amazon Alexa zdůrazňuje kompromisy mezi snadností použití a suverenitou dat. Open-source přístup poskytuje úplnou kontrolu nad ukládáním a zpracováním dat, čímž eliminuje závislost na službách dodavatelů, které mohou být ukončeny.

\vspace{5mm}
\noindent\textbf{Klíčová slova:} chytrá domácnost, MQTT, mTLS, Home Assistant, ESP32, edge computing, IoT bezpečnost, simulace senzorů

% ============================================================================
\newpage
