\chapter{System Architecture}
\label{ch:architecture}

This chapter presents the overall architecture of the implemented smart home system, including network topology, communication protocols, and security design.

% ============================================================================
% 3.1 Architecture Overview
% ============================================================================
\section{Architecture Overview}
\label{sec:arch-overview}

The system follows a distributed architecture with geographically separated components connected through secure communication channels. Figure~\ref{fig:system-arch} illustrates the high-level system architecture.

% TODO: Replace placeholder with actual architecture diagram
\begin{figure}[htbp]
\centering
\fbox{\parbox{0.9\textwidth}{\centering\vspace{3cm}[System Architecture Diagram Placeholder]\\
Include: EMQX (Germany), Home Assistant (Prague), ESP32 nodes, Simulator, iPhone\vspace{3cm}}}
\caption{Overall system architecture showing distributed components.}
\label{fig:system-arch}
\end{figure}

The architecture comprises four main layers:

\begin{enumerate}
    \item \textbf{Sensor Layer:} Physical ESP32/ESP32-S3 sensor nodes and the Python simulator generating virtual device data.
    
    \item \textbf{Communication Layer:} EMQX MQTT broker handling message routing with mTLS encryption.
    
    \item \textbf{Control Layer:} Home Assistant platform providing automation, dashboards, and device management.
    
    \item \textbf{Storage Layer:} InfluxDB time-series database with Grafana visualization.
\end{enumerate}

% ============================================================================
% 3.2 Network Topology
% ============================================================================
\section{Network Topology}
\label{sec:network-topology}

% ----------------------------------------------------------------------------
\subsection{Geographic Distribution}
\label{subsec:geo-distribution}

The system components are distributed across multiple geographic locations to demonstrate real-world deployment scenarios and validate cross-region communication capabilities.

\begin{table}[htbp]
\centering
\caption{Geographic Distribution of System Components}
\label{tab:geo-distribution}
\begin{tabular}{llll}
\toprule
\textbf{Location} & \textbf{Component} & \textbf{Platform} & \textbf{Role} \\
\midrule
Nuremberg, DE & EMQX Broker & VPS (Debian) & Message routing \\
Prague, CZ & Home Assistant & ESXi VM & Control center \\
Prague, CZ & ESP32 Nodes & Physical hardware & Sensor data \\
Prague, CZ & Xiaomi Air Purifier & Commercial device & Integration demo \\
Worldwide & Python Simulator & Any VPS & Load testing \\
Anywhere & iPhone HA App & iOS & Mobile control \\
\bottomrule
\end{tabular}
\end{table}

This distribution validates the system's ability to operate across public internet connections while maintaining security through encrypted communications.

% ----------------------------------------------------------------------------
\subsection{Tailscale VPN Integration}
\label{subsec:tailscale}

For administrative access and non-public services, Tailscale provides a zero-trust mesh VPN based on WireGuard. Key benefits include:

\begin{itemize}
    \item \textbf{NAT Traversal:} Automatic handling of network address translation without port forwarding.
    \item \textbf{Key Rotation:} Automatic WireGuard key rotation every hour.
    \item \textbf{Access Control:} Fine-grained ACL policies for service-level permissions.
\end{itemize}

This approach eliminates the need to expose administrative interfaces to the public internet while maintaining convenient remote access.

% ============================================================================
% 3.3 Communication Protocols
% ============================================================================
\section{Communication Protocols}
\label{sec:comm-protocols}

The system employs multiple communication protocols optimized for different use cases:

\begin{table}[htbp]
\centering
\caption{Communication Protocols by Use Case}
\label{tab:protocols}
\begin{tabular}{llll}
\toprule
\textbf{Protocol} & \textbf{Use Case} & \textbf{Security} & \textbf{Latency} \\
\midrule
MQTT (EMQX) & Cross-network messaging & mTLS (ECC) & Medium \\
Native API & HA $\leftrightarrow$ ESP local control & TLS & Low \\
MiIO & Xiaomi device control & Token auth & Low \\
HTTP/WebSocket & Dashboard access & TLS & Variable \\
\bottomrule
\end{tabular}
\end{table}

A dual-connection architecture is employed for ESP32 nodes, utilizing both Native API for low-latency local control and MQTT for cross-network communication. This approach combines the benefits of direct local connections with the flexibility of broker-based messaging.

% ============================================================================
% 3.4 Security Architecture
% ============================================================================
\section{Security Architecture}
\label{sec:security-arch}

The security architecture implements defense-in-depth principles across multiple layers:

\begin{enumerate}
    \item \textbf{Transport Security:} All MQTT communications use mTLS with ECC certificates (secp256r1 curve), providing mutual authentication and encryption.
    
    \item \textbf{Network Security:} Administrative services are accessible only through Tailscale VPN, implementing zero-trust access control.
    
    \item \textbf{Application Security:} Home Assistant employs token-based authentication with configurable session management.
    
    \item \textbf{Device Security:} ESP32 nodes store credentials in encrypted NVS (Non-Volatile Storage) partitions.
\end{enumerate}

The certificate hierarchy consists of a self-signed Certificate Authority (CA) that issues server certificates for the EMQX broker and client certificates for each ESP32 node and simulator instance. This approach enables granular access control and simplified certificate revocation if a device is compromised.

% TODO: Add Figure showing certificate hierarchy

\subsection{mTLS Implementation}
\label{subsec:mtls-impl}

The mutual TLS configuration ensures that only authorized clients can establish connections to the MQTT broker:

\begin{itemize}
    \item \textbf{Server Authentication:} Clients verify the broker's certificate against the trusted CA.
    \item \textbf{Client Authentication:} The broker verifies each client's certificate, rejecting connections from unknown devices.
    \item \textbf{Access Control:} Certificate Common Name (CN) can be used for topic-level authorization.
\end{itemize}

ECC certificates were selected over RSA due to their superior performance on resource-constrained devices, as detailed in Section~\ref{subsec:ecc-rsa}.
