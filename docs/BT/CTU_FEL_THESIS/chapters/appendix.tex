\chapter{ESPHome Configuration Files}
\label{app:esphome}

This appendix contains excerpts from the ESPHome configuration files used for the ESP32 sensor nodes.

\section{ESP32-S3 Edge Intelligence Node}
\label{app:esp32s3-config}

\begin{verbatim}
# esp32s3.yaml (excerpt)
esphome:
  name: esp32s3-node-a
  platform: ESP32
  board: esp32-s3-devkitc-1

wifi:
  ssid: !secret wifi_ssid
  password: !secret wifi_password

api:
  encryption:
    key: !secret api_key

mqtt:
  broker: mqtt.example.com
  port: 8883
  certificate_authority: !secret ca_cert
  client_certificate: !secret client_cert
  client_key: !secret client_key

sensor:
  - platform: bmp085
    temperature:
      name: "S3 Temperature"
    pressure:
      name: "S3 Pressure"
    address: 0x77
    update_interval: 60s

  - platform: mpu6050
    address: 0x68
    accel_x:
      name: "S3 Accel X"
      id: accel_x
    accel_y:
      name: "S3 Accel Y"
      id: accel_y
    accel_z:
      name: "S3 Accel Z"
      id: accel_z
    update_interval: 100ms

  - platform: template
    name: "Resultant G-Force"
    lambda: |-
      float ax = id(accel_x).state;
      float ay = id(accel_y).state;
      float az = id(accel_z).state;
      return sqrt(ax*ax + ay*ay + az*az);
    update_interval: 100ms
    unit_of_measurement: "g"
\end{verbatim}

\section{ESP32 Environment Sensing Node}
\label{app:esp32-config}

\begin{verbatim}
# esp32.yaml (excerpt)
esphome:
  name: esp32-node-b
  platform: ESP32
  board: esp32dev

sensor:
  - platform: tcs34725
    red_channel:
      name: "Color Red"
    green_channel:
      name: "Color Green"
    blue_channel:
      name: "Color Blue"
    illuminance:
      name: "Illuminance"
    address: 0x29
    update_interval: 10s

  - platform: adc
    pin: GPIO34
    name: "Smoke Sensor"
    id: smoke_sensor
    update_interval: 5s
    filters:
      - multiply: 3.3

binary_sensor:
  - platform: gpio
    pin: GPIO27
    name: "Motion Detected"
    device_class: motion

light:
  - platform: rgb
    name: "Status LED"
    id: status_led
    red: output_red
    green: output_green
    blue: output_blue
    effects:
      - pulse:
          name: "Fast Pulse"
          transition_length: 0.5s
          update_interval: 0.5s

automation:
  - trigger:
      platform: numeric_state
      id: smoke_sensor
      above: 2.5
    action:
      - light.turn_on:
          id: status_led
          effect: "Fast Pulse"
          red: 100%
          green: 0%
          blue: 0%
\end{verbatim}

% ============================================================================

\chapter{Python Simulator Code Excerpts}
\label{app:simulator}

This appendix contains key excerpts from the Python sensor simulator.

\section{Simulator Entry Point}
\label{app:sim-entry}

\begin{verbatim}
# smarthome_sim/simulator.py (excerpt)
class Simulator:
    def __init__(self, config, args):
        self.brokers = config.get('brokers', [])
        self.devices = args.devices
        self.workers = args.workers
        self.qos = args.qos
        
    def run(self):
        with Pool(self.workers) as pool:
            pool.map(self._run_worker, range(self.workers))
            
    def _run_worker(self, worker_id):
        devices_per_worker = self.devices // self.workers
        for i in range(devices_per_worker):
            device_id = worker_id * devices_per_worker + i
            self._simulate_device(device_id)
\end{verbatim}

\section{mTLS Configuration}
\label{app:mtls-config}

\begin{verbatim}
# smarthome_sim/broker.py (excerpt)
def _create_tls_context(self, broker_config):
    import ssl
    context = ssl.SSLContext(ssl.PROTOCOL_TLS_CLIENT)
    context.verify_mode = ssl.CERT_REQUIRED
    
    # Load CA certificate
    context.load_verify_locations(broker_config['ca_file'])
    
    # Load client certificate and key for mTLS
    if 'client_cert' in broker_config:
        context.load_cert_chain(
            certfile=broker_config['client_cert'],
            keyfile=broker_config['client_key']
        )
    
    return context
\end{verbatim}

% ============================================================================

\chapter{Certificate Generation}
\label{app:certs}

This appendix documents the certificate generation process for mTLS.

\section{Certificate Authority}
\label{app:ca-gen}

\begin{verbatim}
# Generate CA private key (ECC)
openssl ecparam -genkey -name secp256r1 -out ca.key

# Generate CA certificate
openssl req -new -x509 -days 3650 -key ca.key -out ca.pem \
  -subj "/CN=SmartHome CA/O=CTU FEL/C=CZ"
\end{verbatim}

\section{Server Certificate}
\label{app:server-cert}

\begin{verbatim}
# Generate server private key
openssl ecparam -genkey -name secp256r1 -out server.key

# Generate certificate signing request
openssl req -new -key server.key -out server.csr \
  -subj "/CN=mqtt.example.com"

# Sign with CA
openssl x509 -req -in server.csr \
  -CA ca.pem -CAkey ca.key -CAcreateserial \
  -out server.pem -days 365
\end{verbatim}

\section{Client Certificate}
\label{app:client-cert}

\begin{verbatim}
# Generate client private key
openssl ecparam -genkey -name secp256r1 -out client.key

# Generate certificate signing request
openssl req -new -key client.key -out client.csr \
  -subj "/CN=esp32-client/OU=IoT Devices"

# Sign with CA
openssl x509 -req -in client.csr \
  -CA ca.pem -CAkey ca.key -CAcreateserial \
  -out client.pem -days 365
\end{verbatim}
